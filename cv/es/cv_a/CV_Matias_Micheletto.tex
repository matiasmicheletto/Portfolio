\documentclass[10pt]{article}

\usepackage[left=1in, right=1in, bottom=2cm]{geometry}
\usepackage[spanish]{babel}
\usepackage[utf8]{inputenc}
\usepackage{array, xcolor}
\usepackage{censor}
\usepackage[urlbordercolor={1 1 1}]{hyperref}
\usepackage{longtable}

\newcolumntype{L}{>{\raggedleft}p{0.14\textwidth}}
\newcolumntype{R}{p{0.8\textwidth}}
\newcommand\VRule{\color{lightgray}\vrule width 0.5pt}


\def\censorData{1} % Censurar datos personales para publicar


\title{\bfseries\huge Currículum Vitae}
\author{\Large Matías J. Micheletto}
\date{Abril 2021}

\begin{document}
\maketitle

\section{Datos personales}
\raggedright
\textbf{Apellido y nombres:} Micheletto, Matías Javier \\
\textbf{Documento de identidad:}
\ifx\censorData\undefined
  34.377.688 \\
\else
	\censor{34.377.688} \\
\fi
\textbf{Lugar y fecha de nacimiento:}
\ifx\censorData\undefined
  	Bahía Blanca, 26 de abril de 1989 \\
\else
	\censor{Bahía Blanca, 26 de abril} de 1989 \\
\fi
\textbf{Nacionalidad:} Argentino \\
\textbf{Estado civil:} Soltero \\
\textbf{Domicilio laboral:}
\ifx\censorData\undefined
  	San Andres 800, DIEC-IIIE UNS-CONICET, Bahía Blanca, CP 8000 \\
\else
	\censor{San Andres 800 Bahía Blanca, CP 8000} \\
\fi
\textbf{Teléfono:}
\ifx\censorData\undefined
  	+54 9 291 459 5181 - \textbf{Fax:} +54 9 291 459 5154 \\
\else
	\censor{ +54 9 291 459 5181 - \textbf{Fax:} +54 9 291 459 5154 } \\
\fi
\textbf{Correo electrónico:} matias.micheletto@uns.edu.ar \\

\section{Formación}
\begin{tabular}{L!{\VRule}R}
2020 & {\bf Universidad Nacional del Sur}\\
	 & Doctor en Ingeniería. \\
	 & Calificación de Tesis: Sobresaliente (10/10). \\[5pt]

2016 & {\bf Universidad Nacional del Sur}\\
	 & Ingeniero Electrónico. \\
	 & Promedio con aplazos: 8.17/10. \\[5pt]

2006 & {\bf Escuela de Agricultura y Ganadería U.N.S.} \\
 	 & Técnico en Producción Agropecuaria.\\
	 & Promedio con aplazos: 7.79/10. \\
\end{tabular}

\section{Docencia}
\begin{tabular}{L!{\VRule}R}
2018 - 2020 & {\bf Asistente de Cátedra} \\
	 & Técnicas Digitales - Diseño de Circuitos Lógicos. \\ [5pt]
	 
2016 - 2018 & {\bf Ayudante A (Graduado)} \\
	 & Técnicas Digitales - Diseño de Circuitos Lógicos. \\
\end{tabular}

\section{Becas}
\begin{longtable}{L!{\VRule}R}
2021 - 2024 & {\bf Beca Interna Postoctoral CONICET} \\
	& \textit{``Toma Inteligente de Decisiones para Agricultura de Precisión basada en Aprendizaje Automatizado''.} \\
	& \textbf{Director:} Dr. Carlos I. Chesñevar. \\
   	& \textbf{Co-Director:} Dr. Juan A. Galantini. \\ [5pt]

\newpage

2016 - 2021 & {\bf Beca Interna Doctoral CONICET} \\
 	 & \textit{``Desarrollo de Sistemas Computacionales para Sustento de Tareas Agropecuarias en el Sudoeste de la Provincia de Buenos Aires''.} \\
	 & \textbf{Director:} Dr. Rodrigo M. Santos. \\
	 & \textbf{Co-Director:} Dr. Juan A. Galantini. \\ [5pt]

2015 - 2016 & {\bf Beca de Investigación PGI-MAyDS} \\
	 & \textit{``Sustentabilidad de Construcciones Civiles''.} \\
	 & \textbf{Director:} Dr. Néstor F. Ortega. \\
	 & Secretaría General de Ciencia y Tecnología U.N.S. Res. CSU-752/15. \\[5pt]

2014 & {\bf Beca de Introducción a la Investigación para Alumnos Avanzados} \\
	 & \textbf{Tema:}\textit{``Planificación Óptima de Sistemas de Tiempo Real en Plataformas Multicore''.} \\
	 & \textbf{Director:} Dr. Javier D. Orozco. \\
	 & Secretaría General de Ciencia y Tecnología U.N.S. \\
	 & Res. CSU-135/14. \\[5pt]

2014 & {\bf Beca Interna de Estímulo al Estudio} \\
	 & Dpto. de Ingeniería Eléctrica y Computadoras U.N.S. \\
	 & Res. CSU-213/14. \\[5pt]

2013 & {\bf Beca Interna de Estímulo al Estudio} \\
	 & Dpto. de Ingeniería Eléctrica y Computadoras U.N.S. \\
	 & Res. CSU-128/13. \\
\end{longtable}

\section{Publicaciones en revistas}
\begin{longtable}{L!{\VRule}R}
2020 & {\bf An IoT-based Infraestructure to Enhance Self-Evacuations in Natural Hazardous Events.} \\
 	 & Mariano Finochietto, Gabriel Eggly, Matías Micheletto, Roger Pueyo Centelles, Roc Messeguer, Sergio Ochoa, Rodrigo Santos, Javier Orozco. \\
 	 & Personal ans Ubiquitous Computing. \\[5pt]

2020 & {\bf Development and Validation of a LiDAR Scanner for 3D Evaluation of Soil Vegetal Coverage.} \\
	 & Matías Micheletto, Luciano Zubiaga, Rodrigo Santos, Juan Galantini, Miguel Cantamutto, Javier Orozco. \\
	 & Electronics, 9(1), 109. \\[5pt]
	 
2019 & {\bf Utilizando UML para el aprendizaje del modelado y diseño de sistemas ciber-físicos.} \\
	 & Leonardo Ordínez, Rodrigo Santos, Gabriel Eggly, Matías Micheletto. \\
	 & IEEE-RITA, 15(1), 50-60. \\[5pt]

2019 & {\bf Scheduling Mandatory-Optional-Time Tasks in Homogeneous Multi-Core Systems with Energy Constraints Using Bio-Inspired Meta-Heuristics.} \\
	 & Matías Micheletto, Rodrigo Santos, Javier Orozco. \\
	 & Journal of Universal Computer Science, 25(4), 390-417. \\[5pt]

2018 & {\bf Flying Real-Time Network to Coordinate Disaster Relief Activities in Urban Areas.} \\
	 & Matías Micheletto, Vinicius Petrucci, Rodrigo Santos, Javier Orozco, Daniel Mosse, Sergio Ochoa, Roc Meseguer. \\
	 & Sensors, 18(5), 1662. \\[5pt]

\newpage	 

2017 & {\bf Real-Time Communication Support for Underwater Acoustic Sensor Networks.} \\
	 & Rodrigo Santos, Javier Orozco, Matías Micheletto, Sergio Ochoa, Roc Meseguer, Pere Milan, Carlos Molina. \\
	 & Sensors, 17(7), 1629. \\	 
\end{longtable}

\section{Premios y distinciones}
\begin{tabular}{L!{\VRule}R}
2021 & {\bf Agrotón 2021} \\
	 & \textbf{Primer premio.} \\
	 & \textbf{AgroTeam:} Matías Micheletto, Alejandro André, Matías Timi, Guido Temperini y Martín Rodríguez. \\
	 & \textbf{Propuesta:} TR4: trazabilidad de productos cárnicos para la industria 4.0. \\[5pt]

2018 & {\bf Agrotón 2018} \\
	 & \textbf{Segundo premio.} \\
	 & \textbf{Equipo A:} Matías Micheletto, Alejandro André, Matías Timi, José Augusto Strick, Franco Tronelli y Guido Temperini. \\
	 & \textbf{Propuesta:} Paquete de Automatización e Información Integral de Soluciones Agropecuarias (PAIISA). \\
\end{tabular}

\section{Cursos de Posgrado}
\begin{longtable}{L!{\VRule}R}
2021 & {\bf ``Técnicas Avanzadas de Computación Evolutiva''.} \\
   & Departamento de Ciencias e Ingeniería de la Computación. \\
   & Dictado por: Dr. Ignacio Ponzoni. \\
   & Duración: 120hs.  Nota obtenida: 10 (diez). \\[5pt]

2020 & {\bf ``Análisis Visual de Grandes Conjuntos de Datos''.} \\
   & Departamento de Ciencias e Ingeniería de la Computación. \\
   & Dictado por: Dra. Silvia Castro y Dra. Ma. Luján Ganuza. \\
   & Duración: 90hs.  Nota obtenida: 10 (diez). \\[5pt]

2020 & {\bf ``Sistemas Colaborativos con Restricciones Temporales''.} \\
   & Departamento de Ingeniería Eléctrica y de Computadoras. \\
   & Dictado por: Dr. Javier Orozco y Dr. Rodrigo Santos. \\
   & Duración: 60hs.  Nota obtenida: 10 (diez). \\[5pt]

2017 & {\bf ``Minería de Datos y Aprendizaje Automatizado''.} \\
   & Departamento de Ciencias e Ingeniería de la Computación. \\
   & Dictado por: Dr. Carlos Chesñevar. \\
   & Duración: 90hs.  Nota obtenida: 10 (diez). \\[5pt]

2017 & {\bf ``Tópicos en Big Data''.} \\
   & Departamento de Ingeniería Eléctrica y Computadoras. \\
   & Dictado por: Dr. Claudio Delrieux. \\
   & Duración: 80hs. Nota obtenida: 10 (diez). \\[5pt]

2017 & {\bf ``Modelos Matemáticos de Simulación en la Investigación Agropecuaria''.} \\
   & Departamento de Agronomía. \\
   & Dictado por: Dr. Juan Galantini. \\
   & Duración: 60hs.  Nota obtenida: 10 (diez). \\[5pt]

2016 & {\bf ``Sistemas Distribuidos de Tiempo Real''.} \\
   & Departamento de Ingeniería Eléctrica y Computadoras. \\
   & Dictado por: Dr. Ricardo Cayssials y Dr. Edgardo Ferro. \\
   & Duración: 60hs. Nota obtenida: 10 (diez). \\[5pt]

2016 & {\bf ``Internet de las Cosas y el Control, un Enfoque Holístico''.} \\
   & Departamento de Ingeniería Eléctrica y Computadoras. \\
   & Dictado por: Dr. Rodrigo Santos y Dr. Sergio Ochoa. \\
   & Duración: 60hs. Nota obtenida: 10 (diez). \\[5pt]
\end{longtable}

\section{Pasantías}
\begin{tabular}{L!{\VRule}R}
2014 & {\bf Pasantía interna Dpto. de Ingeniería Eléctrica y Computadoras U.N.S.} \\
	 & Desarrollo de un Libro Multimedial Interactivo de Mecánica. \\
	 & Duración: 10 meses. \\
\end{tabular}

\section{Participación en congresos}
\begin{longtable}{L!{\VRule}R}
2021 & {\bf International Conference on Computer Science and Intelligent Systems (ICCSIS001 2021)} \\
	& \textit{``Unattended Crowdsensing Method to Monitor the Quality Condition of Dirt Roads''.} \\
	& Matías J. Micheletto, Rodrigo Santos, Sergio Ochoa. \\[5pt]

2021 & {\bf 50º Jornadas Argentinas de Informática} \\
	& Congreso de AgroInformática (CAI 2021) \\
	& \textit{``Desarrollo de una Aplicación Híbrida para Cálculos de Siembra''.} \\
	& Matías J. Micheletto, Grabriel Eggly, Juan P. D'Amico, Santiago J. Crocioni. \\[5pt]

2021 & {\bf 50º Jornadas Argentinas de Informática} \\
	& Simposio Argentino de Informática Industrial e Investigación Operativa (SIIO 2021) \\
	& \textit{``Planificación de flujos con vencimiento en redes definidas por software''.} \\
	& Martín Fraga, Andrés Llinas, Matías J. Micheletto, Paula Zabala, Rodrigo Santos. \\[5pt]

2020 & {\bf 49º Jornadas Argentinas de Informática} \\
	& Simposio Argentino de Educación en Informática (SAEI) \\
	& \textit{``Cipressus: Un Sistema de Gestión de Contenido para el Aprendizaje de Sistemas Digitales''.} \\
	& Matías J. Micheletto. \\[5pt]

2019 & {\bf 13$^{th}$ International Conference on Ubiquitous Computing \& Ambient Intelligence} \\
	 & \textit{``Evacuation Supporting System Network (ESSN) based on IoT Components''.} \\
	 & Gabriel Eggly, José Mariano Finochietto, Matías Micheletto, Rodrigo Santos, Sergio Ochoa, Roc Meseguer and Javier Orozco. \\[5pt]

2019 & {\bf 48º Jornadas Argentinas de Informática} \\
	 & 1º Taller Argentino de Internet de las Cosas (TAIC) \\
	 & \textit{``Internet de las Cosas como Bien Social''.} \\
	 & Gabriel Eggly, Mariano Finochietto, Matías J. Micheletto, Rodrigo Santos. \\[5pt]

2018 & {\bf 47º Jornadas Argentinas de Informática} \\
	 & 10º Congreso de AgroInformática (CAI 2018) \\
	 & \textit{``Diseño e Implementación de un Escáner Lidar para Análisis Tridimensional de Covertura Vegetal de Suelos''.} \\
	 & Matías J. Micheletto, Rodrigo Santos, Luciano Zubiaga, Juan Galantini. \\[5pt]

2017 & {\bf XXV Jornadas de Jóvenes Investigadores} \\
	 & \textit{``Planificación Óptima de un Sistema Multiprocesador de Tiempo Real con Restricciones de Precedencia, Comunicación y Energía''.} \\
	 & Matías J. Micheletto, Alejandro Borghero, Gabriel M. Eggly. \\[5pt]

2017 & {\bf 46º Jornadas Argentinas de Informática} \\
	 & 9º Congreso de AgroInformática (CAI 2017) \\
	 & \textit{``Desarrollo de una Aplicación Móvil para Cálculos de Pulverizaciones Agrícolas''.} \\
	 & Gabriel M. Eggly, Matías J. Micheletto, Juan P. D'Amico, Santiago J. Crocioni. \\[5pt]

2016 & {\bf V Congreso Internacional sobre Cambio Climático y Desarrollo Sustentable} \\
	 & \textit{``Afectación de la Luna en la Medición de la Contaminación Lumínica''.} \\
	 & Luciana C. Lambertucci, Matías J. Micheletto, Jorge A. Starobinsky, Néstor F. Ortega. \\[5pt]

2016 & {\bf 10$^{th}$ International Conference on Ubiquitous Computing \& Ambient Intelligence} \\
	 & \textit{``Scheduling Real-Time Traffic in Underwater Acoustic Wireless Sensor Networks''.} \\
	 & Rodrigo Santos, Javier Orozco, Matías Micheletto, Sergio Ochoa, Roc Meseguer, Pere Millan, Carlos Molina. \\[5pt]

2016 & {\bf 45º Jornadas Argentinas de Informática} \\
	 & 8º Congreso de AgroInformática (CAI 2016) \\
	 & \textit{``Diseño e Implementación de un Registrador de Esfuerzos para Maquinaria Agrícola''.} \\
	 & Matías J. Micheletto, Gabriel M. Eggly, Rodrigo Santos. \\[5pt]

\newpage	 

2015 & {\bf V Brazilian Symposium on Computing Systems Engineering} \\
	 & \textit{``Using bioinspired meta-heuristics to solve reward-based energy-aware mandatory/optional real-time tasks scheduling''.} \\
	 & Matías J. Micheletto, Javier D. Orozco, Rodrigo Santos.  \\[5pt]

2014 & {\bf Simposio Argentino de Sistemas Embebidos} \\
	 & Foro tecnológico 2014 \\
	 & \textit{``Design and Implementation of an Embedded Prototype for Monitoring a Combine Harvester''.} \\
	 & Ana S. Arauz Lozano, Matías J. Micheletto, Leonardo Ordinez, Rodrigo Santos.  \\[5pt]

2012 & {\bf 41º Jornadas Argentinas de Informática} \\
	 & Concurso de Trabajos Estudiantiles U.N.L.P. \\
	 & \textit{``Optimizador de Funciones Multivariadas por Enjambre de Partículas''.} \\
	 & Matías J. Micheletto. \\
\end{longtable}

\section{Evaluación de proyectos}
\begin{tabular}{L!{\VRule}R}
2019 & {\bf FONCyT-ANPCyT} \\
	 & PICT-START UP-2018 \\
	 & Participación como par evaluador de proyectos. \\[5pt]

2017 & {\bf CLEI 2017 / 46 JAIIO} \\
	 & Simposio Latinoamericano de Infraestructura, Hardware y Software 2017. \\
	 & Participación como revisor de trabajos. \\
\end{tabular}

\section{Idiomas}
\begin{tabular}{L!{\VRule}R}
	& {\bf Inglés} \\
	&  Comprensión oral y escrita. Redacción de texto científico y técnico.
\end{tabular}

\end{document}
\grid
